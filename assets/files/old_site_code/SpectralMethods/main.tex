% --------------------------------------------------------------
% Abhi's Standard math Preamble.
% --------------------------------------------------------------
 
% Document packages / layout
\documentclass[12pt]{article}
\usepackage[margin=1in]{geometry} %1 inch margins

% Figure Packages
\usepackage{float}
\usepackage{hyperref}
\usepackage{subcaption}
\usepackage{wrapfig}
\usepackage[export]{adjustbox} %center option in include graphics

% Math Packages
\usepackage{amsmath,amsthm,amssymb,mathrsfs,bm}
\usepackage{mathtools}
\usepackage{commath}
\usepackage{esvect} %For derivatives of vectors \vec{u}' -> \vv{u}'

% Code input 
\usepackage{algorithm}
\usepackage{algpseudocode}
%Manual indentation 
\algdef{SE}[SUBALG]{Indent}{EndIndent}{}{\algorithmicend\ }%
\algtext*{Indent}
\algtext*{EndIndent}
\makeatletter
\newenvironment{breakablealgorithm}
  {% \begin{breakablealgorithm}
   \begin{center}
     \refstepcounter{algorithm}% New algorithm
     \hrule height.8pt depth0pt \kern2pt% \@fs@pre for \@fs@ruled
     \renewcommand{\caption}[2][\relax]{% Make a new \caption
       {\raggedright\textbf{\ALG@name~\thealgorithm} ##2\par}%
       \ifx\relax##1\relax % #1 is \relax
         \addcontentsline{loa}{algorithm}{\protect\numberline{\thealgorithm}##2}%
       \else % #1 is not \relax
         \addcontentsline{loa}{algorithm}{\protect\numberline{\thealgorithm}##1}%
       \fi
       \kern2pt\hrule\kern2pt
     }
  }{% \end{breakablealgorithm}
     \kern2pt\hrule\relax% \@fs@post for \@fs@ruled
   \end{center}
  }
\makeatother

% Quality of Life Packages
\usepackage{enumerate}

\newcommand{\coarse}{\mathcal{G}}
\newcommand{\fine}{\mathcal{F}}

\newtheorem*{lemma*}{Lemma}
\newtheorem*{theorem*}{Theorem}
 
\begin{document}

We all know that the Fourier transform is pretty magical, but it turns out to be
extra special to deal with certain PDEs. As it turns out, since the Fourier
basis gives spectral accuracy for smooth functions, we can approximate the
derivative of functions with spectral accuracy and in $\theta(n\log(n))$ time,
since we have the Fast Fourier Transform. Therefore, as a substitute to finite
difference methods, we can use spectral methods to discretize the spacial
derivative operator.

\section{A Simple Heat Equation}

Let's look at the initial-boundary value problem for the 1D heat equation:

$$
\begin{cases}
  \frac{\partial u}{\partial t} = \alpha \frac{\partial^2 u}{\partial x^2}
  & x \in [0, 2\pi), t \in (0, T] \\
  u(x,0) = u_0(x), & x \in [0, 2\pi) \\
  u(0,t) = u(2\pi, t) = 0
\end{cases}
$$

This is a toy example, where we can directly solve this problem using the
fourier

\section{Code: Spectral Differentiation}

If you would like to look at the code for spectral differentiation, I would have
to reference you to Trefethen's excellent text called \textit{Spectral Methods
in Matlab}; really excellent on the matter. As far as the code I used, take
a look!

\inputminted[fontsize=\footnotesize, linenos]{c}{./code/spectral_diff.h}

\end{document}
